%% -*- mode: latex; mode: flyspell -*-
\documentclass[12pt, letter]{article}

%% Class name and Assignment number
%%
\newcommand{\courseName}{ECE 435 Medical~Image~Processing}
\newcommand{\assignName}{Assignment~4}

%% Packages
\usepackage{amsmath,amsfonts,amssymb,amsthm,dsfont}
\usepackage{graphicx}
\usepackage[bookmarks=false]{hyperref}
\usepackage{color}
\usepackage{lipsum}
\usepackage{amsmath}
\usepackage{booktabs}

%% Paper format
\usepackage{geometry}
\geometry{
    letterpaper,
    %% total={216mm,279mm}, %< NSERC size
    margin=2.00cm,     %< default
    %% margin=1.87cm,       %< NSERC tightest
}

%% Headers and footers
\usepackage[explicit]{titlesec}
\newpagestyle{titlesec_assignment}{
  \sethead{\courseName}{}{\assignName}\setfoot{}{\thepage}{}
  \headrule
  %% \footrule
}

\begin{document}

%% Set header and footer
\pagestyle{titlesec_assignment}

%% Title
\title{\courseName\\\assignName}
\author{Yiping Wang V00894385}
\maketitle

\paragraph{Question 2:} Test the algorithm on the image ``angiogram.tif''. You will provide the following results:
\begin{enumerate}
    \item The binarized image with the value of the threshold at convergence
    \item The histogram of ‘angiogram.tif’ image with the value of the threshold specified on it.
    \item A table containing the value of the threshold at every iteration before reaching convergence.
\end{enumerate}

\paragraph{Answer:}
The following results are obtained when tolerance is set to $\mathbf{10^{-8}}$. The initial estimate of the threshold is randomly selected from the element of the image.  

The binarized image with threshold at convergence of $\mathbf{103.9122}$, the histogram of ``angiogram.tif'' image with the threshold at convergence of $\mathbf{103.9122}$ and the histogram of number of binary values are shown in Figure~\ref{fig:q2-1}. Table~1 shows the value of the threshold at every iteration before reaching convergence.

\begin{figure}
    \centering
    \includegraphics[width=14cm]{without-histogram-equalization.png}
    \caption{(a) Binarized Image; (b) ``angiogram.tif'' Image Histogram when Threshold at Convergence of 103.9122, Red Line Marks the Threshold at Convergence; (c) Histogram of Number of Binary Values}
    \label{fig:q2-1}
\end{figure}

\begin{center}
\begin{tabular}{llr}  
\toprule
\multicolumn{2}{c}{Table 1} \\
\cmidrule(r){1-2}
Iteration   & Threshold Value \\
\midrule
1      & 89 \\
2 & 93.4788 \\
3  & 96.4683 \\
4     & 98.5574 \\
5 & 100.0893 \\
6 & 101.6115 \\
7 & 102.3741 \\
8 & 103.1515 \\
9 & 103.9122 \\
\bottomrule
\label{table:q2-1}
\end{tabular}    
\end{center}

Since the initial estimate of the threshold is randomly selected from the element of the image, the result of converage could be different. 

\paragraph{Question 3:} Apply histogram equalization to ‘angiogram.tif’. You may choose to work with the Matlab \textbf{histeq} function. Next, apply the same thresholding algorithm to the equalized image. Compute the new threshold, and the new binarized images

The binarized image with threshold at convergence of $\mathbf{127.0794}$ and the histogram of ``angiogram.tif'' image with the threshold at convergence of $\mathbf{127.0794}$ shows in Figure~\ref{fig:q3-1}. Table~2 shows the value of the threshold at every iteration before reaching convergence.

\begin{figure}
    \centering
    \includegraphics[width=14cm]{with-histogram-equalization.png}
    \caption{(a) Binarized Image; (b) ``angiogram.tif'' Image Histogram when Threshold at Convergence of 127.0794, Red Line Marks the Threshold at Convergence; (c) Histogram of Number of Binary Values}
    \label{fig:q3-1}
\end{figure}

\begin{center}
\begin{tabular}{llr}  
\toprule
\multicolumn{2}{c}{Table 2} \\
\cmidrule(r){1-2}
Iteration   & Threshold Value \\
\midrule
1 & 28 \\
2 & 78.6122 \\
3 & 102.8717 \\
4 & 115.3512 \\
5 & 121.2035 \\
6 & 125.9473 \\
7 & 127.0794 \\

\bottomrule
\label{table:q2-2}
\end{tabular}    
\end{center}

Since we pre-prcess the image using the Histogram Equalizationto, the threshould of converage always converages to the same number, unlike the case without Histogram Equalization. Moreover, we can see the histogram of number of binary values are roughly the same. 

\paragraph{Question 4: } Compare and discuss the results obtained by optimal thresholding on the original image, and on the image pre-processed with histogram equalization.
\end{document}


%%% Local Variables:
%%% mode: latex
%%% TeX-master: t
%%% End:
